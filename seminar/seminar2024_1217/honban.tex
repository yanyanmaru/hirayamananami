\documentclass{article}
\usepackage[utf8]{inputenc}
\usepackage{amsmath}
\usepackage{amsfonts}
\usepackage{amssymb}
\usepackage{graphicx}
\usepackage{url}
\usepackage{bm}
\usepackage{xcolor}
\usepackage[a4paper,margin=1.0in]{geometry} % 余白を設定
\usepackage{listings}

% このコードで行間を1.5倍にしている 教科書に近くしようと思った。
\renewcommand{\baselinestretch}{1.5}

\title{ゼミ12月17日分}
\author{平山 奈々海}
\date{\today}

\begin{document}

\maketitle


教科書の説明:黒色

\textcolor{blue}{補足:青色}

\textcolor{red}{わからなかったところ、気になったところ:赤色}

\section{最小2乗推定量の性質}

さて、最小2乗推定量 $\bm{\hat{\beta}}$ の平均と共分散行列を求めよう。(命題12.2)
これは確率変数のベクトルについての平均と共分散行列を扱うことになるのでその定義から始める。
$\bm{Z} = (Z_1, \ldots, Z_n)^\top$ を $n$ 個の確率変数を縦に並べたベクトルとし、これを確率ベクトルと呼ぶ。

その平均は $\bm{\mu} = E[\bm{Z}]$、共分散行列は$\bm{\Sigma} = \mathrm{Cov}(\bm{Z}) = E[(\bm{Z} - \bm{\mu})(\bm{Z} - \bm{\mu})^\top]$
で定義される。
平均($\bm{\mu} = E[\bm{Z}]$)を成分で表示すると

\begin{equation*}
  \bm{\mu} =
  \begin{pmatrix}
  \mu_1 \\
  \vdots \\
  \mu_n
  \end{pmatrix}
  =
  \begin{pmatrix}
  E[Z_1] \\
  \vdots \\
  E[Z_n]
  \end{pmatrix}
\end{equation*}
共分散行列は$\sigma_{ij} = E[(Z_i - \mu_i)(Z_j - \mu_j)]$とおくと、


\begin{equation*}
  \bm{\Sigma} = (\sigma_{ij}) =
  \begin{pmatrix}
  E[(Z_1 - \mu_1)^2] & \cdots & E[(Z_1 - \mu_1)(Z_n - \mu_n)] \\
  \vdots & \ddots & \vdots \\
  E[(Z_n - \mu_n)(Z_1 - \mu_1)] & \cdots & E[(Z_n - \mu_n)^2]
  \end{pmatrix}
\end{equation*}
と書けることになる。次の補題は期待値や共分散行列の計算に役立つ。

\bigskip

\newpage

\textbf{補題 12.1} $\bm{Z}$ を $n$ 次元の確率ベクトルで、その平均と共分散行列を $E[\bm{Z}] = \bm{\mu}$, $\mathrm{Cov}(\bm{Z}) = \bm{\Sigma}$ とする。$\bm{A}$ を $r \times n$ 行列, $\bm{b}$ を $r$ 次元ベクトル, $\bm{C}$ を $n \times n$ 行列とすると、次の性質が成り立つ。

\begin{enumerate}
    \item $\bm{A}\bm{Z} + \bm{b}$ の平均は $E[\bm{A}\bm{Z} + \bm{b}] = \bm{A}\bm{\mu} + \bm{b}$ である。
    \item $\bm{A}\bm{Z} + \bm{b}$ の共分散行列は $\mathrm{Cov}(\bm{A}\bm{Z} + \bm{b}) = \bm{A}\bm{\Sigma}\bm{A}^\top$ である。
    \item $E[\bm{Z}^\top \bm{C} \bm{Z}] = \mathrm{tr}(\bm{C}\bm{\Sigma}) + \bm{\mu}^\top \bm{C} \bm{\mu}$ である。ただし $n \times n$ 行列 $\bm{D} = (d_{ij})$ に対して $\mathrm{tr}(\bm{D}) = \sum_{i=1}^n d_{ii}$ で定義される。
\end{enumerate}

\bigskip


\textbf{証明} 

\textcolor{blue}{まず$\bm{A}\bm{Z} + \bm{b}$の成分表示は}


\textcolor{blue}{\begin{equation*}
  \bm{A}\bm{Z} + \bm{b}=  
  \begin{pmatrix}
  a_{11} & \cdots & a_{1n} \\
  \vdots & \ddots & \vdots \\
  a_{r1} & \cdots & a_{rn}
  \end{pmatrix}
  \begin{pmatrix}
    Z_1 \\
    \vdots \\
    Z_n
  \end{pmatrix}
  + 
  \begin{pmatrix}
    b_1 \\
    \vdots \\
    b_r
  \end{pmatrix}
  =
  \begin{pmatrix}
    a_{11} Z_1 + \dots + a_{1n}Z_n \\
    \vdots \\
    a_{r1} Z_1 + \dots + a_{rn}Z_n \\
  \end{pmatrix}
  + 
  \begin{pmatrix}
    b_1 \\
    \vdots \\
    b_r
  \end{pmatrix}
\end{equation*}}

\textcolor{blue}{というr次元ベクトルになる。}

\smallskip



\smallskip

この線形変換した確率ベクトル$\bm{A}\bm{Z} + \bm{b}$の平均は



\begin{align*}
  E[\bm{A}\bm{Z} + \bm{b}] 
  &= 
  \textcolor{blue}{\begin{pmatrix}
    E[a_{11} Z_1 + \dots + a_{1n}Z_n+b_1] \\
    \vdots \\
    E[a_{r1} Z_1 + \dots + a_{rn}Z_n+b_r] \\
  \end{pmatrix}} \\
  &=
  \textcolor{blue}{\begin{pmatrix}
    a_{11}E[Z_1] + \dots + a_{1n}E[Z_n]+b_1 \\
    \vdots \\
    a_{r1}E[Z_1] + \dots + a_{rn}E[Z_n]+b_r\\
  \end{pmatrix}} \\
  &=
  \textcolor{blue}{\begin{pmatrix}
    a_{11} & \cdots & a_{1n} \\
    \vdots & \ddots & \vdots \\
    a_{r1} & \cdots & a_{rn}
    \end{pmatrix}
    \begin{pmatrix}
      E[Z_1] \\
      \vdots \\
      E[Z_n]
    \end{pmatrix}
  +
  \begin{pmatrix}
    b_1 \\
    \vdots \\
    b_r
  \end{pmatrix}}
  \\
  &= \bm{A}E[\bm{Z}] + \bm{b} = \bm{A}\bm{\mu} + \bm{b}
\end{align*}

となる。

\newpage

(2)の共分散行列は、\textcolor{blue}{(1)より$E[\bm{A}\bm{Z} + \bm{b}] = \bm{A}\bm{\mu} + \bm{b}$を用いて}
\begin{align*}
  \bm{Cov}(\bm{A}\bm{Z} + \bm{b}) 
  &= E\{ (\bm{A}\bm{Z} + \bm{b} - (\bm{A}\bm{\mu} + \bm{b})) (\bm{A}\bm{Z} + \bm{b} - (\bm{A}\bm{\mu} + \bm{b}))^\top \} \\
  &= E\{ (\bm{A}(\bm{Z} - \bm{\mu})) (\bm{A}(\bm{Z} - \bm{\mu}))^\top \} \\
  &= \textcolor{blue}{E[ \bm{A}(\bm{Z} - \bm{\mu})(\bm{Z} - \bm{\mu})^\top \bm{A}^\top ]} \\
  &= \bm{A}E[(\bm{Z} - \bm{\mu})(\bm{Z} - \bm{\mu})^\top] \bm{A}^\top \\
  &= \bm{A}\bm{\Sigma}\bm{A}^\top
\end{align*}


となる。

\bigskip

(3) については、\textcolor{blue}{$E[\bm{Z}-\bm{\mu}]=\bm{0}$より}
\begin{align*}
  E[\bm{Z}^\top \bm{C} \bm{Z}]  
  &= E[(\bm{Z} - \bm{\mu}+\bm{\mu})^\top \bm{C} (\bm{Z} - \bm{\mu}+\bm{\mu})] \\
  &= \textcolor{blue}{E[(\bm{Z} - \bm{\mu})^\top \bm{C} (\bm{Z} - \bm{\mu})+(\bm{Z} - \bm{\mu})^\top \bm{C} \bm{\mu}+\bm{\mu}^\top \bm{C} (\bm{Z} - \bm{\mu})+\bm{\mu}^\top \bm{C} \bm{\mu}]}  \\
  &= \textcolor{blue}{E[(\bm{Z} - \bm{\mu})^\top \bm{C} (\bm{Z} - \bm{\mu})]+E[(\bm{Z} - \bm{\mu})^\top \bm{C} \bm{\mu}]+E[\bm{\mu}^\top \bm{C} (\bm{Z} - \bm{\mu})]+E[\bm{\mu}^\top \bm{C} \bm{\mu}]}  \\
  &= E[(\bm{Z} - \bm{\mu})^\top \bm{C} (\bm{Z} - \bm{\mu})] + \bm{\mu}^\top \bm{C} \bm{\mu} 
\end{align*}


さらにトレースの性質 $\mathrm{tr}(\bm{AC}) = \mathrm{tr}(\bm{CA})$ を用いると、
\begin{align*}
  E[(\bm{Z} - \bm{\mu})^\top \bm{C} (\bm{Z} - \bm{\mu})] 
  &= E[\mathrm{tr} \{ (\bm{Z} - \bm{\mu})^\top \bm{C} (\bm{Z} - \bm{\mu}) \}] \quad \text{スカラーより} \\
  &= \textcolor{blue}{E[\mathrm{tr} \{  \underline{\bm{C} (\bm{Z} - \bm{\mu})} \; \underline{(\bm{Z} - \bm{\mu})^\top} \}]}  \quad \text{$\mathrm{tr}(\bm{AC}) = \mathrm{tr}(\bm{CA})$ より}\\
  &= \mathrm{tr}[\bm{C}E\{ (\bm{Z} - \bm{\mu})(\bm{Z} - \bm{\mu})^\top \}] \quad \text{$E[tr(A)]=tr(E[A])$より} \\
  &= \mathrm{tr}(\bm{C}\bm{\Sigma})
\end{align*}




であるから (3) が成り立つ。


\end{document}