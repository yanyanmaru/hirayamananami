\documentclass{article}
\usepackage{amsmath, amssymb, bm}

\begin{document}

さて、最小2乗推定量 $\bm{\beta}$ の平均と共分散行列を求めよう。
これは確率変数のベクトルについての平均と共分散行列を扱うことになるのでその定義から始める。
$\bm{Z} = (Z_1, \ldots, Z_n)^\top$ を $n$ 個の確率変数を縦に並べたベクトルとし、これを確率ベクトルと呼ぶ。
その平均は $\bm{\mu} = \mathbb{E}[\bm{Z}]$、共分散行列は
\[
\bm{\Sigma} = \mathrm{Cov}(\bm{Z}) = \mathbb{E}[(\bm{Z} - \bm{\mu})(\bm{Z} - \bm{\mu})^\top]
\]
で定義される。成分で表示すると、平均は
\[
\bm{\mu} =
\begin{pmatrix}
\mu_1 \\
\vdots \\
\mu_n
\end{pmatrix}
=
\begin{pmatrix}
\mathbb{E}[Z_1] \\
\vdots \\
\mathbb{E}[Z_n]
\end{pmatrix}
\]
であり、また共分散行列は
\[
\sigma_{ij} = \mathbb{E}[(Z_i - \mu_i)(Z_j - \mu_j)]
\]
とおくと。


\[
\bm{\Sigma} = (\sigma_{ij}) =
\begin{pmatrix}
\mathbb{E}[(Z_1 - \mu_1)^2] & \cdots & \mathbb{E}[(Z_1 - \mu_1)(Z_n - \mu_n)] \\
\vdots & \ddots & \vdots \\
\mathbb{E}[(Z_n - \mu_n)(Z_1 - \mu_1)] & \cdots & \mathbb{E}[(Z_n - \mu_n)^2]
\end{pmatrix}
\]

と書けることになる。次の補題は期待値や共分散行列の計算に役立つ。

\bigskip

\noindent
\textbf{補題 12.1} $\bm{Z}$ を $n$ 次元の確率ベクトルで、その平均と共分散行列を $\mathbb{E}[\bm{Z}] = \bm{\mu}$, $\mathrm{Cov}(\bm{Z}) = \bm{\Sigma}$ とする。$\bm{A}$ を $r \times n$ 行列, $\bm{b}$ を $r$ 次元ベクトル, $\bm{C}$ を $n \times n$ 行列とすると、次の性質が成り立つ。

\begin{enumerate}
    \item $\bm{A}\bm{Z} + \bm{b}$ の平均は $\mathbb{E}[\bm{A}\bm{Z} + \bm{b}] = \bm{A}\bm{\mu} + \bm{b}$ である。
    \item $\bm{A}\bm{Z} + \bm{b}$ の共分散行列は $\mathrm{Cov}(\bm{A}\bm{Z} + \bm{b}) = \bm{A}\bm{\Sigma}\bm{A}^\top$ である。
    \item $\mathbb{E}[\bm{Z}^\top \bm{C} \bm{Z}] = \mathrm{tr}(\bm{C}\bm{\Sigma}) + \bm{\mu}^\top \bm{C} \bm{\mu}$ である。ただし $n \times n$ 行列 $\bm{D} = (d_{ij})$ に対して $\mathrm{tr}(\bm{D}) = \sum_{i=1}^n d_{ii}$ で定義される。
\end{enumerate}

\bigskip

\noindent
\textbf{証明} (1)、(2) については、線形変換した確率変数 $\bm{A}\bm{Z} + \bm{b}$ の平均は
\[
\mathbb{E}[\bm{A}\bm{Z} + \bm{b}] = \bm{A}\mathbb{E}[\bm{Z}] + \bm{b} = \bm{A}\bm{\mu} + \bm{b}
\]
となり、共分散行列は
\[
\mathrm{Cov}(\bm{A}\bm{Z} + \bm{b}) = \mathbb{E}\{ (\bm{A}\bm{Z} + \bm{b} - (\bm{A}\bm{\mu} + \bm{b})) (\bm{A}\bm{Z} + \bm{b} - (\bm{A}\bm{\mu} + \bm{b}))^\top \}
\]
\[
= \mathbb{E}\{ (\bm{A}(\bm{Z} - \bm{\mu})) (\bm{A}(\bm{Z} - \bm{\mu}))^\top \}
\]
\[
= \bm{A}\mathbb{E}[(\bm{Z} - \bm{\mu})(\bm{Z} - \bm{\mu})^\top] \bm{A}^\top = \bm{A}\bm{\Sigma}\bm{A}^\top
\]
となる。(3) については、
\[
\mathbb{E}[\bm{Z}^\top \bm{C} \bm{Z}] = \mathbb{E}[(\bm{Z} - \bm{\mu})^\top \bm{C} (\bm{Z} - \bm{\mu})] + \bm{\mu}^\top \bm{C} \bm{\mu}
\]
さらにトレースの性質 $\mathrm{tr}(\bm{AC}) = \mathrm{tr}(\bm{CA})$ を用いると、
\[
\mathbb{E}[(\bm{Z} - \bm{\mu})^\top \bm{C} (\bm{Z} - \bm{\mu})] = \mathbb{E}[\mathrm{tr} \{ (\bm{Z} - \bm{\mu})^\top \bm{C} (\bm{Z} - \bm{\mu}) \}]
\]
\[
= \mathrm{tr}[\bm{C}\mathbb{E}\{ (\bm{Z} - \bm{\mu})(\bm{Z} - \bm{\mu})^\top \}] = \mathrm{tr}(\bm{C}\bm{\Sigma})
\]
であるから (3) が成り立つ。$\qed$


\end{document}
