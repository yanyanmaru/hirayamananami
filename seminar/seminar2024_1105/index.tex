% Options for packages loaded elsewhere
\PassOptionsToPackage{unicode}{hyperref}
\PassOptionsToPackage{hyphens}{url}
\PassOptionsToPackage{dvipsnames,svgnames,x11names}{xcolor}
%
\documentclass[
  letterpaper,
  DIV=11,
  numbers=noendperiod]{scrartcl}

\usepackage{amsmath,amssymb}
\usepackage{iftex}
\ifPDFTeX
  \usepackage[T1]{fontenc}
  \usepackage[utf8]{inputenc}
  \usepackage{textcomp} % provide euro and other symbols
\else % if luatex or xetex
  \usepackage{unicode-math}
  \defaultfontfeatures{Scale=MatchLowercase}
  \defaultfontfeatures[\rmfamily]{Ligatures=TeX,Scale=1}
\fi
\usepackage{lmodern}
\ifPDFTeX\else  
    % xetex/luatex font selection
\fi
% Use upquote if available, for straight quotes in verbatim environments
\IfFileExists{upquote.sty}{\usepackage{upquote}}{}
\IfFileExists{microtype.sty}{% use microtype if available
  \usepackage[]{microtype}
  \UseMicrotypeSet[protrusion]{basicmath} % disable protrusion for tt fonts
}{}
\makeatletter
\@ifundefined{KOMAClassName}{% if non-KOMA class
  \IfFileExists{parskip.sty}{%
    \usepackage{parskip}
  }{% else
    \setlength{\parindent}{0pt}
    \setlength{\parskip}{6pt plus 2pt minus 1pt}}
}{% if KOMA class
  \KOMAoptions{parskip=half}}
\makeatother
\usepackage{xcolor}
\setlength{\emergencystretch}{3em} % prevent overfull lines
\setcounter{secnumdepth}{5}
% Make \paragraph and \subparagraph free-standing
\makeatletter
\ifx\paragraph\undefined\else
  \let\oldparagraph\paragraph
  \renewcommand{\paragraph}{
    \@ifstar
      \xxxParagraphStar
      \xxxParagraphNoStar
  }
  \newcommand{\xxxParagraphStar}[1]{\oldparagraph*{#1}\mbox{}}
  \newcommand{\xxxParagraphNoStar}[1]{\oldparagraph{#1}\mbox{}}
\fi
\ifx\subparagraph\undefined\else
  \let\oldsubparagraph\subparagraph
  \renewcommand{\subparagraph}{
    \@ifstar
      \xxxSubParagraphStar
      \xxxSubParagraphNoStar
  }
  \newcommand{\xxxSubParagraphStar}[1]{\oldsubparagraph*{#1}\mbox{}}
  \newcommand{\xxxSubParagraphNoStar}[1]{\oldsubparagraph{#1}\mbox{}}
\fi
\makeatother


\providecommand{\tightlist}{%
  \setlength{\itemsep}{0pt}\setlength{\parskip}{0pt}}\usepackage{longtable,booktabs,array}
\usepackage{calc} % for calculating minipage widths
% Correct order of tables after \paragraph or \subparagraph
\usepackage{etoolbox}
\makeatletter
\patchcmd\longtable{\par}{\if@noskipsec\mbox{}\fi\par}{}{}
\makeatother
% Allow footnotes in longtable head/foot
\IfFileExists{footnotehyper.sty}{\usepackage{footnotehyper}}{\usepackage{footnote}}
\makesavenoteenv{longtable}
\usepackage{graphicx}
\makeatletter
\def\maxwidth{\ifdim\Gin@nat@width>\linewidth\linewidth\else\Gin@nat@width\fi}
\def\maxheight{\ifdim\Gin@nat@height>\textheight\textheight\else\Gin@nat@height\fi}
\makeatother
% Scale images if necessary, so that they will not overflow the page
% margins by default, and it is still possible to overwrite the defaults
% using explicit options in \includegraphics[width, height, ...]{}
\setkeys{Gin}{width=\maxwidth,height=\maxheight,keepaspectratio}
% Set default figure placement to htbp
\makeatletter
\def\fps@figure{htbp}
\makeatother

\KOMAoption{captions}{tableheading}
\makeatletter
\@ifpackageloaded{caption}{}{\usepackage{caption}}
\AtBeginDocument{%
\ifdefined\contentsname
  \renewcommand*\contentsname{目次}
\else
  \newcommand\contentsname{目次}
\fi
\ifdefined\listfigurename
  \renewcommand*\listfigurename{図一覧}
\else
  \newcommand\listfigurename{図一覧}
\fi
\ifdefined\listtablename
  \renewcommand*\listtablename{表一覧}
\else
  \newcommand\listtablename{表一覧}
\fi
\ifdefined\figurename
  \renewcommand*\figurename{図}
\else
  \newcommand\figurename{図}
\fi
\ifdefined\tablename
  \renewcommand*\tablename{表}
\else
  \newcommand\tablename{表}
\fi
}
\@ifpackageloaded{float}{}{\usepackage{float}}
\floatstyle{ruled}
\@ifundefined{c@chapter}{\newfloat{codelisting}{h}{lop}}{\newfloat{codelisting}{h}{lop}[chapter]}
\floatname{codelisting}{コード}
\newcommand*\listoflistings{\listof{codelisting}{コード一覧}}
\usepackage{amsthm}
\theoremstyle{plain}
\newtheorem{theorem}{定理}[section]
\theoremstyle{remark}
\AtBeginDocument{\renewcommand*{\proofname}{証明}}
\newtheorem*{remark}{注釈}
\newtheorem*{solution}{解答}
\newtheorem{refremark}{注釈}[section]
\newtheorem{refsolution}{解答}[section]
\makeatother
\makeatletter
\makeatother
\makeatletter
\@ifpackageloaded{caption}{}{\usepackage{caption}}
\@ifpackageloaded{subcaption}{}{\usepackage{subcaption}}
\makeatother

\ifLuaTeX
\usepackage[bidi=basic]{babel}
\else
\usepackage[bidi=default]{babel}
\fi
\babelprovide[main,import]{japanese}
% get rid of language-specific shorthands (see #6817):
\let\LanguageShortHands\languageshorthands
\def\languageshorthands#1{}
\ifLuaTeX
  \usepackage{selnolig}  % disable illegal ligatures
\fi
\usepackage{bookmark}

\IfFileExists{xurl.sty}{\usepackage{xurl}}{} % add URL line breaks if available
\urlstyle{same} % disable monospaced font for URLs
\hypersetup{
  pdftitle={11月5日 7-4},
  pdflang={ja},
  colorlinks=true,
  linkcolor={blue},
  filecolor={Maroon},
  citecolor={Blue},
  urlcolor={Blue},
  pdfcreator={LaTeX via pandoc}}


\title{11月5日 7-4}
\author{}
\date{2024-11-05}

\begin{document}
\maketitle

\renewcommand*\contentsname{目次}
{
\hypersetup{linkcolor=}
\setcounter{tocdepth}{3}
\tableofcontents
}

\section{2項分布}\label{ux9805ux5206ux5e03}

\(X \sim Bin(n,p)\)とする。

まず二項分布における\(p\)
の推定量を\(\displaystyle \hat{p}=\frac{X}{n}\) とおく。
\(\displaystyle E_p[\hat{p}] = \frac{E_p[X]}{n}=\frac{np}{n}=p\)
より不偏推定量である。

ちなみに不偏推定量の定義は、以下である。

\begin{theorem}[]\protect\hypertarget{thm-line}{}\label{thm-line}

\(\hat{\theta}\) が不偏推定量であるとは

\[
E_\theta[\hat{\theta}(X)] = \theta, \quad \delta\theta
\]

が成り立つことである。

\end{theorem}

不偏推定量 \(\displaystyle \hat{p}=\frac{X}{n}\)
がUMVUであることを示すには、次の等式を満たす。

\begin{theorem}[クラメル・ラオの不等式を用いたUMVUの証明]\protect\hypertarget{thm-line}{}\label{thm-line}

不偏推定量\(\hat{\theta^*}\) が

\[
Var_\theta[\hat{\theta^*}] = \frac{1}{I_n(\theta)},\quad \delta\theta
\]

を満たせば、\(\hat{\theta*}\)はUMVUである。

\end{theorem}

つまり、\(\displaystyle Var_\theta \left[\hat{p}\right] = \frac{1}{I_n(\theta)}\)
であることを示したい。

まず、推定量の分散は\(\displaystyle Var[\hat{p}]=\frac{Var[X]}{n^2}=\frac{p(1-p)}{n}\)
である。

次にフィッシャー情報量\(I_n(\theta)\)を求めたい。

\begin{theorem}[フィッシャー情報量の求め方]\protect\hypertarget{thm-line}{}\label{thm-line}

\(\ell'(\theta,X)\)は対数充度関数\(\ell(\theta,X)\) を\(\theta\)
で微分したものである。

\[
I_n(\theta) = E_\theta \left[(\ell'(\theta,X)^2)\right]
\]

\end{theorem}

2項分布の確率変数 \(f(x,p)=p^x(1-p)^{n-x}\begin{pmatrix}
n  \\
x  \\
\end{pmatrix}\)の対数を\(p\) で微分すると

\[
\begin{aligned}
    \ell'(p, x) &= \frac{\partial}{\partial p} \left( x \log p + (n - x) \log (1 - p) + \log \binom{n}{x} \right) \\
    &= \frac{x}{p}+\frac{-(n-x)}{1-p} = \frac{x-np}{p(1-p)}
\end{aligned}
\]

となる。従ってフィッシャー情報量は

\[
I(p)=E[\ell'(p,X)^2]=\frac{E[(X-np)^2]}{(p(1-p))^2} =\frac{np(1-p)}{(p(1-p))^2} = \frac{n}{p(1-p)}
\]

となり、これは\(\displaystyle \frac{1}{Var_p[\hat{p}]}\)
に一致する。したがって\(\hat{p}\)は\(UMVU\)であることが確かめられた。

\(E[(X-np)^2]\)はテクニカル!

\section{正規分布}\label{ux6b63ux898fux5206ux5e03}

次に正規分布の母平均\(\mu\)の推定において標本平均\(\bar{X}\)
がUMVUであることを示す。

まず、\(\mu\) に関するフィッシャー情報量を求める。

\[
\ell(\mu,x)= - \frac{(x-\mu)^2}{2\sigma^2} - \frac{1}{2} \log(2\pi \sigma^2)
\]

を\(\mu\)
で偏微分すると\(\displaystyle \ell'(\mu,x)=\frac{x-\mu}{\sigma^2}\)
を得る。したがって、

\[
I(\mu)=\frac{E[(X-\mu)^2]}{\sigma^4} = \frac{1}{\sigma^2}
\]

となり、\(\bar{X}\) がUMVUであることが示された。

\(E[(X-\mu)^2]\)もテクニカル!




\end{document}
