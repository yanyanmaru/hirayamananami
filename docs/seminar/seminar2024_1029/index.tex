% Options for packages loaded elsewhere
\PassOptionsToPackage{unicode}{hyperref}
\PassOptionsToPackage{hyphens}{url}
\PassOptionsToPackage{dvipsnames,svgnames,x11names}{xcolor}
%
\documentclass[
  letterpaper,
  DIV=11,
  numbers=noendperiod]{scrartcl}

\usepackage{amsmath,amssymb}
\usepackage{iftex}
\ifPDFTeX
  \usepackage[T1]{fontenc}
  \usepackage[utf8]{inputenc}
  \usepackage{textcomp} % provide euro and other symbols
\else % if luatex or xetex
  \usepackage{unicode-math}
  \defaultfontfeatures{Scale=MatchLowercase}
  \defaultfontfeatures[\rmfamily]{Ligatures=TeX,Scale=1}
\fi
\usepackage{lmodern}
\ifPDFTeX\else  
    % xetex/luatex font selection
\fi
% Use upquote if available, for straight quotes in verbatim environments
\IfFileExists{upquote.sty}{\usepackage{upquote}}{}
\IfFileExists{microtype.sty}{% use microtype if available
  \usepackage[]{microtype}
  \UseMicrotypeSet[protrusion]{basicmath} % disable protrusion for tt fonts
}{}
\makeatletter
\@ifundefined{KOMAClassName}{% if non-KOMA class
  \IfFileExists{parskip.sty}{%
    \usepackage{parskip}
  }{% else
    \setlength{\parindent}{0pt}
    \setlength{\parskip}{6pt plus 2pt minus 1pt}}
}{% if KOMA class
  \KOMAoptions{parskip=half}}
\makeatother
\usepackage{xcolor}
\setlength{\emergencystretch}{3em} % prevent overfull lines
\setcounter{secnumdepth}{-\maxdimen} % remove section numbering
% Make \paragraph and \subparagraph free-standing
\makeatletter
\ifx\paragraph\undefined\else
  \let\oldparagraph\paragraph
  \renewcommand{\paragraph}{
    \@ifstar
      \xxxParagraphStar
      \xxxParagraphNoStar
  }
  \newcommand{\xxxParagraphStar}[1]{\oldparagraph*{#1}\mbox{}}
  \newcommand{\xxxParagraphNoStar}[1]{\oldparagraph{#1}\mbox{}}
\fi
\ifx\subparagraph\undefined\else
  \let\oldsubparagraph\subparagraph
  \renewcommand{\subparagraph}{
    \@ifstar
      \xxxSubParagraphStar
      \xxxSubParagraphNoStar
  }
  \newcommand{\xxxSubParagraphStar}[1]{\oldsubparagraph*{#1}\mbox{}}
  \newcommand{\xxxSubParagraphNoStar}[1]{\oldsubparagraph{#1}\mbox{}}
\fi
\makeatother


\providecommand{\tightlist}{%
  \setlength{\itemsep}{0pt}\setlength{\parskip}{0pt}}\usepackage{longtable,booktabs,array}
\usepackage{calc} % for calculating minipage widths
% Correct order of tables after \paragraph or \subparagraph
\usepackage{etoolbox}
\makeatletter
\patchcmd\longtable{\par}{\if@noskipsec\mbox{}\fi\par}{}{}
\makeatother
% Allow footnotes in longtable head/foot
\IfFileExists{footnotehyper.sty}{\usepackage{footnotehyper}}{\usepackage{footnote}}
\makesavenoteenv{longtable}
\usepackage{graphicx}
\makeatletter
\def\maxwidth{\ifdim\Gin@nat@width>\linewidth\linewidth\else\Gin@nat@width\fi}
\def\maxheight{\ifdim\Gin@nat@height>\textheight\textheight\else\Gin@nat@height\fi}
\makeatother
% Scale images if necessary, so that they will not overflow the page
% margins by default, and it is still possible to overwrite the defaults
% using explicit options in \includegraphics[width, height, ...]{}
\setkeys{Gin}{width=\maxwidth,height=\maxheight,keepaspectratio}
% Set default figure placement to htbp
\makeatletter
\def\fps@figure{htbp}
\makeatother

\KOMAoption{captions}{tableheading}
\makeatletter
\@ifpackageloaded{caption}{}{\usepackage{caption}}
\AtBeginDocument{%
\ifdefined\contentsname
  \renewcommand*\contentsname{目次}
\else
  \newcommand\contentsname{目次}
\fi
\ifdefined\listfigurename
  \renewcommand*\listfigurename{図一覧}
\else
  \newcommand\listfigurename{図一覧}
\fi
\ifdefined\listtablename
  \renewcommand*\listtablename{表一覧}
\else
  \newcommand\listtablename{表一覧}
\fi
\ifdefined\figurename
  \renewcommand*\figurename{図}
\else
  \newcommand\figurename{図}
\fi
\ifdefined\tablename
  \renewcommand*\tablename{表}
\else
  \newcommand\tablename{表}
\fi
}
\@ifpackageloaded{float}{}{\usepackage{float}}
\floatstyle{ruled}
\@ifundefined{c@chapter}{\newfloat{codelisting}{h}{lop}}{\newfloat{codelisting}{h}{lop}[chapter]}
\floatname{codelisting}{コード}
\newcommand*\listoflistings{\listof{codelisting}{コード一覧}}
\usepackage{amsthm}
\theoremstyle{plain}
\newtheorem{theorem}{定理}[section]
\theoremstyle{remark}
\AtBeginDocument{\renewcommand*{\proofname}{証明}}
\newtheorem*{remark}{注釈}
\newtheorem*{solution}{解答}
\newtheorem{refremark}{注釈}[section]
\newtheorem{refsolution}{解答}[section]
\makeatother
\makeatletter
\makeatother
\makeatletter
\@ifpackageloaded{caption}{}{\usepackage{caption}}
\@ifpackageloaded{subcaption}{}{\usepackage{subcaption}}
\makeatother

\ifLuaTeX
\usepackage[bidi=basic]{babel}
\else
\usepackage[bidi=default]{babel}
\fi
\babelprovide[main,import]{japanese}
% get rid of language-specific shorthands (see #6817):
\let\LanguageShortHands\languageshorthands
\def\languageshorthands#1{}
\ifLuaTeX
  \usepackage{selnolig}  % disable illegal ligatures
\fi
\usepackage[]{biblatex}
\usepackage{bookmark}

\IfFileExists{xurl.sty}{\usepackage{xurl}}{} % add URL line breaks if available
\urlstyle{same} % disable monospaced font for URLs
\hypersetup{
  pdftitle={10月29日 不偏分散がカイ二乗分布に従う話},
  pdflang={ja},
  colorlinks=true,
  linkcolor={blue},
  filecolor={Maroon},
  citecolor={Blue},
  urlcolor={Blue},
  pdfcreator={LaTeX via pandoc}}


\title{10月29日 不偏分散がカイ二乗分布に従う話}
\author{}
\date{2024-10-29}

\begin{document}
\maketitle

\renewcommand*\contentsname{目次}
{
\hypersetup{linkcolor=}
\setcounter{tocdepth}{3}
\tableofcontents
}

\section{不偏分散が自由度n-1のカイ二乗分布に従う}\label{ux4e0dux504fux5206ux6563ux304cux81eaux7531ux5ea6n-1ux306eux30abux30a4ux4e8cux4e57ux5206ux5e03ux306bux5f93ux3046}

\subsection{証明したいこと}\label{ux8a3cux660eux3057ux305fux3044ux3053ux3068}

\begin{theorem}[Line]\protect\hypertarget{thm-line}{}\label{thm-line}

The equation of any straight line, called a linear equation, can be
written as:

\[
y = mx + b
\]

\end{theorem}

See 定理~\ref{thm-line}.

\subsection{概要}\label{ux6982ux8981}

\(G\) の具体例の一つとして、Helmert行列と呼ばれるものがある。

\[
G=
\begin{bmatrix}
  \frac{1}{\sqrt{n}} & \frac{1}{\sqrt{n}} &\frac{1}{\sqrt{n}} &\dots &\frac{1}{\sqrt{n}}\\
   \frac{1}{\sqrt{1 \cdot 2}} & \frac{-1}{\sqrt{1 \cdot 2}}& 0 & \dots & 0\\
\frac{1}{\sqrt{2 \cdot 3}} & \frac{1}{\sqrt{2 \cdot 3}}& \frac{-2}{\sqrt{2 \cdot 3}}& \dots & 0\\
\dots &\dots&\dots& \ddots &\dots\\
\frac{1}{\sqrt{(n-1) \cdot n}}& \frac{1}{\sqrt{(n-1) \cdot n}}& 
\frac{1}{\sqrt{(n-1) \cdot n}}& \dots & \frac{1-n}{\sqrt{(n-1) \cdot n}}& \\
\end{bmatrix}
\]

これを使って不偏分散(標本分散)がどのような分布に従うかを考える。

標本分散は\(s^2 = \frac{1}{n}\sum_{i=1}^n (X_i-\bar{X})^2\)
、不偏分散は\(V^2 =\frac{1}{n-1}\sum_{i=1}^n (X_i-\bar{X})^2\)である。

不偏分散の式変形について考えると、

\[
\sum_{i=1}^n (X_i-\bar{X})^2 = \sum_{i=1}^n X_i^2-n\bar{X}^2=\sum_{i=1}^n Y_i^2-Y_1^2=\sum_{i=2}^n Y_i^2
\]

\(Y\)が標準多変量正規分布に従い、\(Y_i\)
はそれぞれ独立に標準正規分布に従う。標準正規分布の二乗和はカイ二乗分布に従うことを知っているので、\(\sum_{i=1}^n (X_i-\bar{X})^2\) は自由度\(n-1\)のカイ二乗分布に従う。このことから、\(ns^2,(n-1)V^2\)は自由度\(n-1\)
のカイ二乗分布に従う。

\subsection{細かい説明}\label{ux7d30ux304bux3044ux8aacux660e}

\(G\)
として1行目の要素が全て等しく、\((\frac{1}{\sqrt{n}}, \dots,\frac{1}{\sqrt{n}})\)
となっているものを考える。

\(g_1\) に対して、\(g_1, \dots,g_n\) が、\(R^n\)
の正規直交底をなすように、\(g_2, \dots,g_n\) を適当に選び、\(G\)
の各行とすれば良い。

\(G\) の一つの具体的な取り方として、\(G\)
の2行目を\(\frac{(1,-1,0, \dots,0)}{\sqrt{2}}\)
とし、3行目を\(\frac{(1,1,-2,0, \dots,0)}{\sqrt{6}}\)
とする。以下同様に第\(k\)行を

\[
\frac{\left( \overbrace{1, \dots, 1}^{k-1 \text{個}}, -k + 1, 0, \dots, 0 \right)}{\sqrt{k(k-1)}}
\]

とおく。このとき\(G\) の各行は正規直交ベクトル系をなし、\(G\)
は直交行列となる。このとき\(G\) を用いた変換はHelmert変換と呼ばれる。

\(G\) は直交行列であるから、

\[
\sum_{i=1}^n X_i^2 = X^{\top}X=(GX)^{\top}GX = Y^{\top} Y = \sum_{i=1}^n Y_i^2
\]

\begin{quote}
1つ目の等式の補足

\(X = (X_1, \dots,X_n)^{\top}\)
より、\(X^{\top}X = (X_1, \dots,X_n)(X_1, \dots,X_n)^{\top}=\sum_{i=1}^n X_i^2\)
\end{quote}

\begin{quote}
2つ目の等式の補足

\(X^{\top}X=(GX)^{\top}GX\)は\(G\)が直交行列なので\(GG^{\top}=I_n\)
である。\(Y\) に変換できるような形に変えている。
\end{quote}

また\(G\)の第1行の選び方により、\(Y_1=\sqrt{n}\bar{X}\)
である。(第1行が\((\frac{1}{\sqrt{n}}, \dots,\frac{1}{\sqrt{n}})\)
より)

\[
\sum_{i=1}^n (X_i - \bar{X})^2= \sum_{i=1}^n X_i^2 - n \bar{X}^2 = \sum_{i=1}^n Y_i^2 - Y_1^2 = \sum_{i=2}^n Y_i^2
\]

\begin{quote}
1つ目の等式の補足

\(\sum_{i=1}^n (X_i - \bar{X})^2= \sum_{i=1}^n X_i^2 - n \bar{X}^2\)
はどっかで出てきたと思うけど、展開したら計算できるはず。
\end{quote}

\begin{quote}
2つ目の等式の補足

\(Y_1=\sqrt{n}\bar{X}\) より。これがおそらくHelmert行列を使った理由。
\(GX=G(X_1,X_2,\dots, X_n)^{\top} = \begin{bmatrix}
\frac{1}{\sqrt{n}} & \frac{1}{\sqrt{n}} &\frac{1}{\sqrt{n}} &\dots &\frac{1}{\sqrt{n}}\\
\frac{1}{\sqrt{1 \cdot 2}} & \frac{-1}{\sqrt{1 \cdot 2}}& 0 & \dots & 0\\
\frac{1}{\sqrt{2 \cdot 3}} & \frac{1}{\sqrt{2 \cdot 3}}& \frac{-2}{\sqrt{2 \cdot 3}}& \dots & 0\\
\dots &\dots&\dots& \ddots &\dots\\
\frac{1}{\sqrt{(n-1) \cdot n}}& \frac{1}{\sqrt{(n-1) \cdot n}}&
\frac{1}{\sqrt{(n-1) \cdot n}}& \dots & \frac{1-n}{\sqrt{(n-1) \cdot n}}& \\
\end{bmatrix}(X_1,X_2,\dots, X_n)^{\top}\)
\end{quote}

したがって\(ns^2=\sum_{i=1}^n (X_i - \bar{X})^2= \sum_{i=2}^n Y_i^2\)
は自由度\(n-1\)のカイ二乗分布に従う。(Yが標準多変量正規分布に従ってて、それぞれは独立に標準正規分布に従うから。正規分布の二乗和はカイ二乗分布に従うという性質から。自由度が\(n-1\)なのは\(Y_i\)
が\(n-1\)個だから)

また、\(Y_1, Y_2 , \dots, Y_n\)
は互いに独立に標準正規分布に従うため、\(Y_2, \dots, Y_n\)
は\(Y_1=\sqrt{n}\bar{X}\)
と互いに独立である。\(ns^2= \sum_{i=2}^n Y_i^2\) も\(\bar{X}\)
と独立である。以上で\(\bar{X}\)と\(s^2\)
が互いに独立であり、\(\sigma^2 \ne 1\)
の場合を含めて考えれば、\(\frac{ns^2}{\sigma^2}=\frac{\sum_{i=1}^n (X_i - \bar{X})^2}{\sigma^2}\)
が自由度\(n-1\)のカイ二乗分布に従うことが示された。

ついでに、\(Y_1=\sqrt{n}\bar{X}\) であることと、\(Y_1\)
が標準正規分布に従うことを用いると、標本平均\(\bar{X}\)の分布もわかる。

\(\bar{X} \sim N(0,\frac{1}{n})\) になるため、\(\mu = 0, \sigma^2 = 1\)
ではない場合も含めると、\(\bar{X} \sim N(\mu,\frac{\sigma^2}{n})\) になることもわかる。

\begin{quote}
\(\sigma^2 \ne 1\) の場合を含めて考えるの意味

~これまで証明してきた\(X_1,X_2, \dots, X_n\) は
\(\mu = 0, \sigma^2 = 1\) の時である。

~\(\frac{\sum_{i=1}^n (X_i - \bar{X})^2}{\sigma^2}=\sum_{i=1}^n (\frac{X_i - \bar{X}}{\sigma})^2= \sum_{i=1}^n (\frac{X_i-\mu}{\sigma}-\frac{\mu-\bar{X}}{\sigma})^2 = \sum_{i=1}^n (Z_i-\bar{Z})^2\)\\
\strut ~よって、\(\sigma^2\) で割っている。
\end{quote}


\printbibliography



\end{document}
